% Created 2015-02-13 Fri 23:32
\documentclass[11pt]{article}
\usepackage[latin1]{inputenc}
\usepackage[T1]{fontenc}
\usepackage{fixltx2e}
\usepackage{graphicx}
\usepackage{longtable}
\usepackage{float}
\usepackage{wrapfig}
\usepackage{soul}
\usepackage{textcomp}
\usepackage{marvosym}
\usepackage{wasysym}
\usepackage{latexsym}
\usepackage{amssymb}
\usepackage{hyperref}
\tolerance=1000
\providecommand{\alert}[1]{\textbf{#1}}

\title{OS Lab 05}
\author{Anurag Shirolkar (120050003) Dheerendra Rathor (120050033)}
\date{\today}
\hypersetup{
  pdfkeywords={},
  pdfsubject={},
  pdfcreator={Emacs Org-mode version 7.9.3f}}

\begin{document}

\maketitle

\setcounter{tocdepth}{3}
\tableofcontents
\vspace*{1cm}
\section{Question 1}
\label{sec-1}

\begin{enumerate}
\item 
\begin{enumerate}
\item where to put the user program : in the directory src/user
\item execute following commands
\begin{itemize}
\item make (in the build directory)
\item geek
\item <name of the program> (in the geek terminal)
\end{itemize}
\item the header files in the directory \textbf{include/libc} can be used in the user programs
\end{enumerate}
\item 
\begin{enumerate}
\item what is the purpose?

\begin{center}
\begin{tabular}{ll}
 file       &  purpose                                                \\
\hline
 syscall.h  &  declarations of all the systemcalls                    \\
 syscall.c  &  definition of the systemcalls declared in syscall.h    \\
 conio.h    &  declaration of all the console input/output functions  \\
 conio.c    &  definitions of the functions declared in conio.h       \\
\end{tabular}
\end{center}


\item to add new syscall 
      add the syscall to the enum which has list of all the syscalls
      write the declaration in the syscall.h file
      write the definition in the syscall.c file
\end{enumerate}
\item In the wrapper of the syscall function the last argument is of the form \\
\textbf{SYSCALL\_{}REGS_<i>}   \\
     where i is the number of parameters in the system call
\item There is a structure \textbf{kernel\_{}thread} defined in the file \textbf{include/geekos/kthread.h}
     that structure has a member \textbf{int pid}. The current\_{}thread variable in the syscall.c
     file is the pointer to the structure of current thread which contains the pid member.
\begin{enumerate}
\item 
\end{enumerate}
\end{enumerate}
\section{Question 2}
\label{sec-2}

  In the geek terminal Type the following command : \\
  \textbf{\$ q2} \\
  The input will be read till \textbf{@} and printed to the console.
  
  
\section{Question 3}
\label{sec-3}

  In the geek terminal Type the following command : \\
  \textbf{\$ q3}\\
  produces the following output
\begin{quote}
Output from old get time of day syscall : <time> \\
Output from new get time of day syscall : <time>
\end{quote}
\section{Question 4}
\label{sec-4}

  Created a new member \textbf{sys\_{}call\_{}count} in the kthread.h
  which is initialized as 0 in kthread.c. Whenever the syscall handler
  in \textbf{trap.c} is called the \textbf{sys\_{}call\_{}count} of the \textbf{CURRENT\_{}THREAD}
  variable is incremented. Defined a new syscall which returns the count
  of syscalls from CURRENT\_{}THREAD.\\
  Similar method used for file count.

\end{document}
